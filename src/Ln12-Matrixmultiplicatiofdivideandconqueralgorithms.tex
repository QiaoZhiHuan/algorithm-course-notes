\chapter{分治算法之矩阵乘法}
\begin{introduction}
	\item 问题引入
	\item 用分治法解决
\end{introduction}
\section{问题引入}
在线性代数中我们学习过矩阵乘法,已知规模为 $n$ 的矩阵 $A$ 和 $B$,我们可以直接计算行列积算法的复杂度,单次数乘的复杂度我们已知是$ O(1)$每次计算行列积需要进行$n$次乘法,总共需要计算$n^2$次行列积,所以不难得知基于行列积的矩阵乘法复杂度为$O(n^3)$
如果我们尝试分治策略,结果会怎么样呢?首先我们将矩阵$C = AB$进行二分。
$$
A=\left[\begin{array}{ll}
A_{11} & A_{12} \\
A_{21} & A_{22}
\end{array}\right] B=\left[\begin{array}{ll}
B_{11} & B_{12} \\
B_{21} & B_{22}
\end{array}\right] C=\left[\begin{array}{ll}
C_{11} & C_{12} \\
C_{21} & C_{22}
\end{array}\right]
$$

如此我们便可得到$C$四个成分的公式:
$$
\begin{array}{l}
C_{11}=A_{11} B_{11}+A_{12} B_{21} \\
C_{12}=A_{11} B_{12}+A_{12} B_{22} \\
C_{21}=A_{21} B_{11}+A_{22} B_{21} \\
C_{22}=A_{21} B_{12}+A_{22} B_{22}
\end{array}
$$
我们运用主方法(master method)通过复杂度递归式计算这个分治策略的 复杂度:
$$
T(n)=8 T\left(\frac{n}{2}\right)+O\left(n^{2}\right)
$$
得到复杂度为$O(n^3)$,这有点令人沮丧,我们的复杂度并没有变得更好,但 是使用类似分支:乘法章节中的策略,我们是否可以通过矩阵加法替代矩阵 乘法,来减少一点分治递归树展开的速度呢?下面我们介绍 stressen 算法, 这种算法将8次矩阵乘法转换成了7次矩阵乘法,代价则是大量的加减操 作和中间矩阵。
$$
\begin{array}{c}
P_{1}=A_{11}\left(B_{12}-B_{22}\right) \\
P_{2}=\left(A_{11}+A_{12}\right) B_{22} \\
P_{3}=\left(A_{21}+A_{22}\right) B_{11} \\
P_{4}=A_{22}\left(B_{21}-B_{11}\right) \\
P_{5}=\left(A_{11}+A_{22}\right)\left(B_{11}+B_{22}\right) \\
P_{7}=\left(A_{11}+A_{21}\right)\left(B_{11}+B_{12}\right)
\end{array}
$$
得到了上面的中间矩阵之后,我们便可以通过组合得到最终结果
$$
C_{11}=P_{5}+P 4-P_{2}+P_{6}\\
C_{12}=P_{1}+P_{2} \\
C_{21}=P_{3}+P_{4} \\
C_{22}=P_{5}+P_{1}-P_{3}-P_{7}
$$
从而我们得到递归式
$$
T(n)=7 T\left(\frac{n}{2}\right)+O\left(n^{2}\right)
$$

由主方法可得复杂度为 $n^{lg 7}$
PS:虽然 stressen 方法的时间复杂度很美好,但是空间复杂度非常高,因 为生成了大量的中间矩阵,实际运行在计算机上时,在矩阵阶数很大时表现 较好(n>30),但在较小的阶数表现还不如原始的矩阵乘法;所以我们在分 治递归中可以使用这种策略:设定一个阶数 m,当矩阵阶数 n>m 时使用 stressen 算法进行分治;当 n<=m 时使用原始的矩阵乘法。
